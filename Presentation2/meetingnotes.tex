% Created 2016-02-14 Sun 14:05
\documentclass[11pt]{article}
\usepackage[utf8]{inputenc}
\usepackage[T1]{fontenc}
\usepackage{fixltx2e}
\usepackage{graphicx}
\usepackage{longtable}
\usepackage{float}
\usepackage{wrapfig}
\usepackage{rotating}
\usepackage[normalem]{ulem}
\usepackage{amsmath}
\usepackage{textcomp}
\usepackage{marvosym}
\usepackage{wasysym}
\usepackage{amssymb}
\usepackage{hyperref}
\tolerance=1000
\author{Nooreen S Dabbish, Kaijun Wang, Dr. Deep}
\date{\textit{<2016-02-14 Sun>}}
\title{Skype Meeting Notes}
\hypersetup{
  pdfkeywords={},
  pdfsubject={},
  pdfcreator={Emacs 24.4.1 (Org mode 8.2.10)}}
\begin{document}

\maketitle
Assignments/instructions from Dr. Deep are in \textbf{bold}.

\section{math.sort}
\label{sec-1}

Kaijun: number is how many parts you want
\begin{itemize}
\item checks only the first factor by default
\item \url{http://www.inside-r.org.packages/cran/psych/docs/mat.sort}

\begin{itemize}
\item K: we want one or two clusters, so it regresses the columns of the
\end{itemize}
covariance matrix

\begin{itemize}
\item How can I use the top three or four? (It sorts by the first factor by default)
\item fa( )

\item \textbf{Select f=1 and f=2 and display the correlation matrices}
\item Does the structure change?

\item f=2, first two or only the second?
\item K: it groups the columns

\item \textbf{dowload the function and unpack that function}
\end{itemize}
\end{itemize}


\section{Lasso}
\label{sec-2}
\begin{itemize}
\item huge number of predictors (15,576)
\item we want to identiify which correlations are changing with 
male and female
\item lasso simultaneously picks variables and determines importance

\item \textbf{Map the non-zero back to the brain}

\item Matrix vectorization and adding response
\item save all the data in a three D matrix X$_{\text{ijk}}$ X[i,j,k]
 X[i, , ,] --> correlation matrix of ith invdividual
c(X[i,,])
upper.tri()

\item \textbf{Streamline data as .Rda} Y-vector and 3-d X matrix

\item save(Y,X,file="<>.Rdata'')

\item \textbf{Any difference between lasso logistic and lasso linear}
  Are the selected coefficients different?
\end{itemize}

\section{Paper to present on 2/22}
\label{sec-3}

\begin{itemize}
\item highly cited, very recent paper

\item Main focus: one part is lasso logistic and linear regression
 Problem of how to select the threshold? The method we are doing
takes all the correlation as a variable in the model. This method is
not scalable for large dimentsional brain data with hundreds or thousands
\end{itemize}
of sensors. It is not an intelligent way for tackling the spatial correlation
between nodes.

\section{Kaijun: on graph kernel}
\label{sec-4}

\begin{itemize}
\item graph kernel is the relationship beween two graphs
\item for example random walks gives a short path kernel
\item the two-sample test: actuallyy tests the

\item \textbf{More insight into definition of the kernel} I just want to know the formula.

\item What people are doing right now: lasso approach, kernel approach. Will the
\end{itemize}
kernel approach help us to classify? How does it help us to answer the
real question, if I have a Y vector and want to predict that?

\begin{itemize}
\item K: we look at the means and if they are the same, the graphs are the same.

\item Kernel methods answers an important question of whehter the groups
\end{itemize}
have a different graph. But can you use this method to predict? If I reject and say 
they are different, the next step would be to build a classifier and tell
us whether it is y=1 or y=2.

\begin{itemize}
\item *First, two-sample hypothesis testing. Second, if I reject that hypothesis (there is a difference)
\end{itemize}
I have to build a classifier.*

\begin{itemize}
\item The other method was lasso logistic, Nooreen can predict whether it is
\end{itemize}
male or female. We have to build that capacity for kernel. Nooreen answered the
second problem without answering the first question.

\begin{itemize}
\item Kaijun is developing a test statistic. It is only meaningful to build a
\end{itemize}
classifier if the answer is reject.

\begin{itemize}
\item write down these questions and mention them in your 2/22 presentation.
\end{itemize}
We want to develop a single method that can answer both questions simultaneously.

\begin{itemize}
\item \textbf{Kaijun assignment: give me some numbers for the application of this method}
\end{itemize}

\section{Summary}
\label{sec-5}
First point will be the two interesting questions 1, 2, 
The method to attack the first problem (5 min, Kaijun) will be the graph kernel part.
Then (5 min, Nooreen) on the prediction lasso part.

Both of us present the paper (10 min, both)
Last (10 min, focus on multiple testing thresholding) \textbf{new} understand that lasso logistic is not scalable. It
is not the smartest way to attach that problem. We have to bring graph 
into the picture. What is the threshold that we should pick?

 What is the right way to go from correlation to binary adjacency 
matrix? The paper will help you take a step in that direction.

 You have all your correlations vectorized, take the Fisher's Z transformation
 1/2 of log r \ldots{} After taking this transformation the numbers are very close
to normal with variance $\frac{1}{(n-3)}$, so I can compute the P-values for 
all the correlation. 

Now the multiple testing problem is waiting for us. We will find a cutoff and get
a binary matrix 1 and 0.
\begin{itemize}
\item will send code for that
\item this is an attempt at calculated the adjacency matrix

\item the problem will be isolated node.
\end{itemize}
% Emacs 24.4.1 (Org mode 8.2.10)
\end{document}